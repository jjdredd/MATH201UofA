% MATH 201 Lab notes (c) by Carlos Contreras And Philippe Gaudreau
% MATH 201 Lab notes is licensed under a 
% Creative Commons Attribution 4.0 International license.
% CC BY 4.0

% You should have received a copy of the license along with this
% work. If not, see <http://creativecommons.org/licenses/by/4.0/>.

\documentclass[11pt]{article}
% MATH 201 Lab notes (c) by Carlos Contreras And Philippe Gaudreau
% MATH 201 Lab notes is licensed under a 
% Creative Commons Attribution 4.0 International license.
% CC BY 4.0

% You should have received a copy of the license along with this
% work. If not, see <http://creativecommons.org/licenses/by/4.0/>.

%% libraries
\usepackage[utf8x]{inputenc}
\usepackage{xcolor}
\usepackage[left=1.5cm,right=1.5cm,top=2.0cm,bottom=1.5cm,headheight=110pt]{geometry}
\usepackage{amsmath}
\usepackage{amssymb}
\usepackage{graphicx}
\usepackage{xifthen}
\usepackage{sverb}
\usepackage{fancyhdr}
\usepackage{mdframed}
\usepackage{textcomp}

%%%%%%%%%%%%%%%%%%%%%%%%%%%%%%%%%%%%%%%%%%%%%%%%%%%%%%%%%%%%%%%%%%%%%%%
% PDF compiling
\usepackage{ifpdf}
\ifpdf %
        \DeclareGraphicsExtensions{.pdf}%
\else %
        \DeclareGraphicsExtensions{.eps,.ps}%
\fi

%%%%%%%%%%%%%%%%%%%%%%%%%%%%%%%%%%%%%%%%%%%%%%%%%%%%%%%%%%%%%%%%%%%%%%%
% Figures path
\graphicspath{{figures/}}

%%%%%%%%%%%%%%%%%%%%%%%%%%%%%%%%%%%%%%%%%%%%%%%%%%%%%%%%%%%%%%%%%%%%%%%
% Problem counter
\newcounter{Problem}
\setcounter{Problem}{0}

%%%%%%%%%%%%%%%%%%%%%%%%%%%%%%%%%%%%%%%%%%%%%%%%%%%%%%%%%%%%%%%%%%%%%%%
% Definitions
\def\LabSolutions{\clearpage \newpage \begin{center} {\Large \it Solutions} \end{center} \setcounter{Problem}{0}}
\def\QuizSolutions{\newpage \begin{center} {\Large \it Solutions} \end{center} \setcounter{Problem}{0}}
\def\degree{\textdegree}
\def\grade#1{\begin{flushright} {\small [#1]}\\ \end{flushright} \vspace{-10pt}}
\def\codecolor{red!50!black}
\def\code#1{\textcolor{\codecolor}{\tt #1}}
\def\examname#1{%
                \ifnum\value{page}>1%
                    \newpage%
                \else%
                    \vspace*{5pt}%
                \fi%
                \large \textbf{#1} \setcounter{Problem}{0}\vspace{10pt}}
\def\topic#1{\par\needspace{2\baselineskip} \noindent \textsl{\footnotesize #1}}

 
%%%%%%%%%%%%%%%%%%%%%%%%%%%%%%%%%%%%%%%%%%%%%%%%%%%%%%%%%%%%%%%%%%%%%%%
% Environments
\newenvironment{problem}%
     {\stepcounter{Problem}%
      \begin{list}{\textbf{\arabic{Problem}}.~}{}%
      \item}%
     {\end{list}\vspace*{5pt}}

\newenvironment{solution}%
     {\indent \textit{Solution} \newline}%
     {\begin{flushright}$\blacksquare$\end{flushright}}

\newenvironment{preamble}%
     {\vspace*{1em}\begin{mdframed}[leftmargin=1cm,rightmargin=1cm]}%
     {\end{mdframed}\vspace*{1em}}

\newenvironment{multchoice}%
     {\begin{enumerate} \addtolength{\leftskip}{2em} \renewcommand{\labelenumi}{(\alph{enumi})}}
     {\end{enumerate}}

\newenvironment{formulaitem}%
     {\setlength{\leftmargini}{1.5em}\begin{itemize}%
      \setlength\itemindent{-\itemindent}%
      \renewcommand{\labelitemi}{$\rightarrow$}}%
     {\end{itemize}}


%%%%%%%%%%%%%%%%%%%%%%%%%%%%%%%%%%%%%%%%%%%%%%%%%%%%%%%%%%%%%%%%%%%%%%%
% New theorems
\newtheorem{theorem}{Theorem}


\makeatletter

%%%%%%%%%%%%%%%%%%%%%%%%%%%%%%%%%%%%%%%%%%%%%%%%%%%%%%%%%%%%%%%%%%%%%%%
%% New commands
\newcommand*{\course}[1]{\gdef\@course{#1}}
\newcommand*{\coursecode}[1]{\gdef\@coursecode{#1}}
\newcommand*{\term}[1]{\gdef\@term{#1}}
\newcommand*{\instructor}[1]{\gdef\@instructor{#1}}
\newcommand*{\lqnumber}[1]{\gdef\@lqnumber{#1}}
\newcommand*{\labtitle}[1]{\gdef\@labtitle{#1}}
\newcommand*{\quizversion}[1]{\gdef\@quizversion{#1}}
\newcommand*{\probleminfo}[1]{\noindent \textsl{\footnotesize #1}}

% Title header for labs
\newcommand\makelabtitle{%
  \begin{flushleft}%
  {\scshape \@coursecode~\@course~-- University of Alberta}\\%
  {\scshape \@term~-- Labs -- \@instructor}\\%
  {\scshape Authors: Carlos Contreras, Philippe Gaudreau, Rahim Usubov}%
  \end{flushleft}%
  \begin{center}%
  {\Large \bf \@lqnumber:~\@labtitle}%
  \end{center}%
  \thispagestyle{empty}%
  \global\let\@course\@empty%
  \global\let\@labtitle\@empty%
}

% Title header for quizzes
\newcommand\makequiztitle{%
  \begin{flushleft}%
  {\scshape \@coursecode~\@course~-- University of Alberta}\\%
  {\scshape \@term~-- Labs -- \@instructor}\\%
  \end{flushleft}%
  \begin{center}%
  {\Large \bf \@lqnumber} \marginpar{\tiny\tt [\@quizversion]}%
  \end{center}%
  \thispagestyle{empty}%
  \global\let\@course\@empty%
  \global\let\@quizversion\@empty%
}

%%%%%%%%%%%%%%%%%%%%%%%%%%%%%%%%%%%%%%%%%%%%%%%%%%%%%%%%%%%%%%%%%%%%%%%
% Fancy header package
\fancyhead[L]{\small {\scshape \@coursecode~-- \@lqnumber~-- \@term~-- \@instructor}}
\pagestyle{fancy}

\makeatother


\usepackage{hyperref}
\usepackage{cancel}
\usepackage{booktabs}


\begin{document}

\course{Differential Equations}
\coursecode{MATH 201}
\term{Winter 2018}
\instructor{Carlos Contreras, Rahim Usubov}
\lqnumber{Lab 7}
\labtitle{Laplace transform}
\makelabtitle

\topic{Inverse Laplace transform}

\begin{problem}
Determine the inverse Laplace transform of
\[\frac{s-1}{2s^{2}+4s+6}.\]
\end{problem}

\begin{problem}
Determine ${\cal L}^{-1}\{ {\it F }\}$ 
\begin{equation*}
s^2 F(s) +sF(s)-6F(s) = \dfrac{s^2+4}{s^2+s}.
\end{equation*}
\end{problem}


\begin{problem}
Find the inverse Laplace transform of 
\[\tanh^{-1} (s).\]
\end{problem}

\topic{Solving IVP using LT}
\begin{problem}
Solve the given initial value problem using the method of Laplace transforms
\begin{equation*}
y^{\prime \prime} - y^{\prime}-2y =0 ; \quad y(0)=-2, \quad y^{\prime}(0) = 5.
\end{equation*}
\end{problem}


\begin{problem}
Solve for $Y(s)$, the Laplace transform of the solution $y(t)$ to the given initial value problem
\begin{equation*}
y^{\prime \prime} +4y = g(t) ; \quad y(0)=-1, \quad y^{\prime}(0) = 0,
\end{equation*}
where
\begin{equation*}
g(t) = \bigg\{ \begin{array}{ccc}
t, \quad t<2 \\
5, \quad t>2
\end{array}.
\end{equation*}
\end{problem}

\begin{problem}
Solve the given initial value problem using the method of Laplace transforms
\begin{equation*}
y^{\prime \prime} + 3 t y^{\prime} - 6y = 1 ; \quad y(0)=0, \quad y^{\prime}(0) = 0.
\end{equation*}
\end{problem}



% 
% \begin{problem}
% Find solutions to the given initial value problem
% \begin{equation*}
% y^{\prime \prime} +ty^{\prime}-y = 0 ; \quad y(0)=0, \quad y^{\prime}(0) = 3.
% \end{equation*}
% \end{problem}








%%%%%%%%%%%%%%%%%%%%%%%%%%%%%%%%%%%%%%%%%%%%%%%%%%%%%%%%%%%%%%%%%%%%%%%%%%%%%%%%%%%%%%%%%%%%%%%%%%%



\LabSolutions


Theory and problems from: Nagel, Saff \& Sneider, \textit{Fundamentals of Differential Equations}, Eighth Edition, Adisson--Wesley.

\begin{preamble}
 
\begin{formulaitem}
\item Definition and properties of \textbf{Laplace Transform. }
% \begin{table}[h]
\renewcommand{\arraystretch}{1.5}
\setlength{\tabcolsep}{10pt}
\begin{center}
\begin{tabular}{|l|}
\hline
$\mathcal{L}\{f\}(s)=F(s)=\int_{0}^{\infty}f(t)e^{-st}dt$\\
$\mathcal{L}\{f+g\}=F(s)+G(s)$ \\
$\mathcal{L}\{cf\}=cF(s)$ \\
$\mathcal{L}\{e^{at}f\}=F(s-a)$ \\
$\mathcal{L}\{f'\}=sF(s)-f(0)$ \\
$\mathcal{L}\{f''\}=s^{2}F(s)-sf(0)-f'(0)$ \\
$\mathcal{L}\{t^{n}f\}=(-1)^{n}\frac{d^{n}}{ds^{n}}F(s)$ \\ 
\hline
\end{tabular}
\end{center}
% \caption{Definition and properties of Laplace Transform.}
% \end{table}

\item Brief \textbf{table of Laplace Transforms}.
% \begin{table}[h]
\renewcommand{\arraystretch}{1.5}
\setlength{\tabcolsep}{10pt}
\begin{center}
\begin{tabular}{|ll|}
\hline
$f(t)$ & $F(s)=\mathcal{L}\{f\}(s)$ \\
\hline
$1$ & $\frac{1}{s}$ \\
$e^{at}$ & $\frac{1}{s-a} \quad s>a$\\
$t^{n}$ & $\frac{n!}{s^{n+1}}$ \\
$\sin bt$ & $\frac{b}{s^{2}+b^{2}}$ \\
$\cos bt$ & $\frac{s}{s^{2}+b^{2}}$ \\
$e^{at}t^{n}$ & $\frac{n!}{(s-a)^{n+1}} \quad s>a$ \\
$e^{at}\sin bt$ & $\frac{b}{(s-a)^{2}+b^{2}} \quad s>a$ \\
$e^{at}\cos bt$ & $\frac{s-a}{(s-a)^{2}+b^{2}} \quad s>a$ \\
\hline 
\end{tabular}
\end{center}
% \caption{Brief table of Laplace Transforms.}
% \end{table}

\item \textbf{Laplace transform of an IVP}. Let the IVP
\[ay''+by' +cy = f(t),\quad y(0)=y_{0}, \quad y'(0)=y'_{0}.\]
Then the Laplace transform of the solution $y(t)$ is
\[Y(s)= \underbrace{\frac{(as+b)y_{0}+ay'_{0}}{as^{2}+bs+c}}_{\text{initial conditions}}+\underbrace{\frac{F(s)}{as^{2}+bs+c}}_{\text{particular sol.}}.\]

\end{formulaitem}


\end{preamble}

\newpage




\begin{problem}
Determine the inverse Laplace transform of
\[\frac{s-1}{2s^{2}+4s+6}.\]
\end{problem}
\begin{solution}
We focus on the denominator. The first attempt in inverse Laplace transforms is to factorize the denominator. However, in this case $2s^{2}+4s+6$ has complex roots. The next attempt is to complete squares to get something of the form $(s-a)^{2}+b^{2}$.
\[2s^{2}-4s +6 = 2(s^{2}+2s+3)=2((s+1)^{2}+2)=2((s+1)^{2}+(\sqrt{2})^{2}).\]
The $2$ in front of the polynomial is nothing but a constant. 
Now, we rewrite the numerator so we have terms of the form \[c\frac{s-a}{(s-a)^{2}+b^{2}},\qquad\text{and}\qquad c\frac{b}{(s-a)^{2}+b^{2}}.\]
That is
\begin{align*}
\mathcal{L}^{-1}\left\{\frac{s-1}{2s^{2}+4s+6}\right\}&= \mathcal{L}^{-1}\left\{\frac{s+1-2}{2((s+1)^{2}+(\sqrt{2})^{2})}\right\}\\
&=\mathcal{L}^{-1}\left\{\frac{1}{2}\frac{s+1}{(s+1)^{2}+(\sqrt{2})^{2}}+ \frac{-\sqrt{2}}{2}\frac{\sqrt{2}}{(s+1)^{2}+(\sqrt{2})^{2}}\right\}.
\end{align*}
Thus, it follows from the table of Laplace transforms that
\[\boxed{\mathcal{L}^{-1}\left\{\frac{s-1}{2s^{2}+4s+6}\right\}=\frac{1}{2}e^{-t}\cos\sqrt{2}t-\frac{\sqrt{2}}{2}e^{-t}\sin\sqrt{2}t}.\]
\end{solution}




\begin{problem}
Determine ${\cal L}^{-1}\{ {\it F }\}$ 
\begin{equation*}
s^2 F(s) +sF(s)-6F(s) = \dfrac{s^2+4}{s^2+s}.
\end{equation*}
\end{problem}
\begin{solution}
Isolating $F(s)$
\begin{eqnarray*}
s^2 F(s) +sF(s)-6F(s) & = & \dfrac{s^2+4}{s^2+s} \\
F(s)(s^2+s-6) & = & \dfrac{s^2+4}{s^2+s} \\
F(s) & = & \left(\dfrac{1}{s^2+s-6}\right) \left(\dfrac{s^2+4}{s^2+s}\right) \\
 & = &  \dfrac{s^2+4}{s(s+1)(s+3)(s-2)}\\
 & = &  \dfrac{A}{s} + \dfrac{B}{s+1} + \dfrac{C}{s+3} + \dfrac{D}{s-2} \\
\end{eqnarray*}
If we combine these fractions under the same denominator, we have:
\begin{eqnarray*}
F(s)=\dfrac{A(s+1)(s+3)(s-2)+B(s)(s+3)(s-2)+C(s)(s+1)(s-2)+D(s)(s+1)(s+3)}{s(s+1)(s+3)(s-2)}  \\
\end{eqnarray*}
Collecting all the powers of s we find:
\begin{eqnarray*}
F(s)=\dfrac{s^3 (A+B+C+D) + s^2 (2A+B-C+4D) + s(-5A-6B-2C+3D)-6A }{s(s+1)(s+3)(s-2)}  \\
\end{eqnarray*}

This implies that:
\begin{eqnarray}
A+B+C+D & = & 0 \\
2A+B-C+4D & = & 1\\
-5A-6B-2C+3D & = & 0\\
-6A & = & 4
\end{eqnarray}

In Matrix form we have:
\begin{equation*}
 \left[ \begin {array}{ccccc} 1&1&1&1&0\\ \noalign{\medskip}2&1&-1&4&1
\\ \noalign{\medskip}-5&-6&-2&3&0\\ \noalign{\medskip}-6&0&0&0&4
\end {array} \right]
\end{equation*}
Which upon reducing, we obtain:
\begin{equation*}
 \left[ \begin {array}{ccccc} 1&0&0&0&- \dfrac{2}{3}\\ \noalign{\medskip}0&1&0&0
&\dfrac{5}{6}\\ \noalign{\medskip}0&0&1&0&-{\dfrac {13}{30}}
\\ \noalign{\medskip}0&0&0&1&{\dfrac {4}{15}}\end {array} \right]
\end{equation*}

Hence ,
\begin{eqnarray*}
F(s)  & = & -\dfrac{2}{3} \dfrac{1}{s} +\dfrac{5}{6}\dfrac{1}{s+1} -\dfrac{13}{30} \dfrac{1}{s+3} + \dfrac{4}{15}\dfrac{1}{s-2} \\
\end{eqnarray*}

Taking the inverse Laplace transform, we have:
\begin{equation}
\boxed{f(t) = {\cal L}^{-1}\{ {\it F }\} = -\dfrac{2}{3} + \dfrac{5}{6} e^{-t} -\dfrac{13}{30} e^{-3t} + \dfrac{4}{15} e^{2t}}.
\end{equation}
\end{solution}




\begin{problem}
Find the inverse Laplace transform of 
\[\tanh^{-1} (s).\]
\end{problem}
\begin{solution}
In problems of inverse Laplace transform of trigonometric or trascendental functions that are not in the table, usually one of its derivatives is a fraction. In this case
\[(\tanh^{-1}s)'=\frac{1}{1-s^{2}},\]
which looks like the Laplace transform of $\sinh$. For this problem we will use the property
\[\mathcal{L}\{t^{n}f(t)\}=(-1)^{n}F^{(n)}(s).\]
Let $F(s)=\tanh^{-1}(s)$, then
\[-F'(s)=-\frac{1}{1-s^{2}}=\frac{1}{s^{2}-1}.\]
Using the property
\[tf(t) =\mathcal{L}^{-1}\left\{\frac{1}{s^{2}-1}\right\}=\sinh t.\]
Isolating $f(t)$, we have
\[\boxed{f(t)=\frac{\sinh t}{t}}\]
\end{solution}




\begin{problem}
Solve the given initial value problem using the method of Laplace transforms
\begin{equation*}
y^{\prime \prime} - y^{\prime}-2y =0 ; \quad y(0)=-2, \quad y^{\prime}(0) = 5.
\end{equation*}
\end{problem}
\begin{solution}
Applying a Laplace transform on both sides, we obtain:
\begin{equation*}
{\cal L}^{}\{ y^{\prime \prime} \} -{\cal L}^{}\{ y^{\prime } \} - 2 {\cal L}^{}\{ y \} =0
\end{equation*}
\begin{equation*}
\left(s^2 Y(s) - s y(0)-y^{\prime}(0)\right) -\left(sY(s)-y(0)\right) - 2 Y(s) =0
\end{equation*}
\begin{equation*}
s^2 Y(s) + 2s -5 -sY(s)-2 - 2 Y(s) =0
\end{equation*}
Isolating $Y(s)$, we obtain:

\begin{equation*}
Y(s) = \dfrac{-2s+7}{s^2-s-2} = \dfrac{-2s+7}{(s+1)(s-2)}
\end{equation*}

We will now have to decompose $\dfrac{-2s+7}{(s+1)(s-2)}$ into partial fractions:

\begin{eqnarray*}
\dfrac{-2s+7}{(s+1)(s-2)} & = & \dfrac{A}{s+1} + \dfrac{B}{s-2} \\
 & = & \dfrac{A(s-2)+B(s+1)}{(s+1)(s-2)}  \\
  & = & \dfrac{s(A+B)+(-2A+B)}{(s+1)(s-2)}
\end{eqnarray*}

Hence, we have:

\begin{eqnarray*}
A+B & = & -2 \\
-2A+B & = & 7
\end{eqnarray*}
Solving this system, we find $A=-3$ and $B=1$.
Hence,

\begin{equation*}
Y(s) = -\dfrac{3}{s+1} + \dfrac{1}{s-2}
\end{equation*}

Taking the inverse Laplace transform, we find

\begin{equation*}
\boxed{y(t)= -3 e^{-t} +e^{2t}}.
\end{equation*}
\end{solution}




\begin{problem}
Solve for $Y(s)$, the Laplace transform of the solution $y(t)$ to the given initial value problem
\begin{equation*}
y^{\prime \prime} +4y = g(t) ; \quad y(0)=-1, \quad y^{\prime}(0) = 0,
\end{equation*}
where
\begin{equation*}
g(t) = \bigg\{ \begin{array}{ccc}
t, \quad t<2 \\
5, \quad t>2
\end{array}
\end{equation*}
\end{problem}
\begin{solution}
Applying a Laplace transform on both sides, we obtain:

\begin{equation*}
{\cal L}^{}\{ y^{\prime \prime} \} +4 {\cal L}^{}\{ y \} = {\cal L}^{}\{ g(t) \}
\end{equation*}
To evaluate ${\cal L}\{g(t)\}$ we need to write $g(t)$ in term of unit step functions,

\[g(t)=t+(5-t)u(t-2)\]

We know ${\cal L}\{f(t-a)u(t-a)\}=F(s)e^{-as}$, but $(5-t)u(t-2)$ does not have that exact form. To solve this we evaluate

\[h(t)=5-t, \quad \Rightarrow \quad h(t+2)=5-(t+2)=3-t,\] 

take Laplace transform
\[{\cal L}\{h(t+2)\}= {\cal L}\{3-t\}=\frac{3}{s}-\frac{1}{s^{2}},\quad \Rightarrow \quad {\cal L}\{h(t)u(t-2)\}=\left( \frac{3}{s}-\frac{1}{s^{2}} \right)e^{-2s}.\]

and simply write

\[G(s)=\frac{1}{s^{2}}+ \left( \frac{3}{s}-\frac{1}{s^{2}} \right)e^{-2s}.\]


% There is no easy way to evaluate ${\cal L}^{}\{ g(t) \} $. We need to resort to the definition:

% \begin{eqnarray*}
% {\cal L}^{}\{ g(t)\} &  =  & \int_{0}^{\infty} e^{-st}g(t) {\rm d}t\\
% & = & \int_{0}^{2} e^{-st}t {\rm d}t + \int_{2}^{\infty} 5 e^{-st} {\rm d}t \\
% & = & \left[ -\dfrac{t}{s} e^{-st} \right]_{0}^{2} +\dfrac{1}{s}  \int_{0}^{2} e^{-st} {\rm d}t + 5 \left[ -\dfrac{1}{s} e^{-st} \right]_{2}^{\infty}\\
% & = & \left[ -\dfrac{t}{s} e^{-st} \right]_{0}^{2} +\dfrac{1}{s} \left[ -\dfrac{1}{s} e^{-st} \right]_{0}^{2} + 5 \left[ -\dfrac{1}{s} e^{-st} \right]_{2}^{\infty}\\
% & = &  -\dfrac{2}{s} e^{-2s} -\dfrac{1}{s^2}(e^{-2s}-1) + \dfrac{5}{s} e^{-2s} \\
% & = &  \dfrac{3e^{-2s}}{s} -\dfrac{e^{-2s}}{s^2} + \dfrac{1}{s^2} \\
% \end{eqnarray*}
Putting all of this together we have

\begin{equation*}
s^2 Y(s) - s y(0)-y^{\prime}(0) +4 Y(s) = \frac{1}{s^{2}}+ \left( \frac{3}{s}-\frac{1}{s^{2}} \right)e^{-2s}
\end{equation*}


\begin{equation*}
s^2 Y(s) + s +4 Y(s) =\frac{1}{s^{2}}+ \frac{3s-1}{s^{2}} e^{-2s}
\end{equation*}
Isolating $Y(s)$, we obtain:

\begin{equation*}
\boxed{Y(s) = \dfrac{3e^{-2s}}{s(s^2+4)} -\dfrac{e^{-2s}}{s^2(s^2+4)} + \dfrac{1}{s^2(s^2+4)} -\dfrac{s}{s^2+4}}.
\end{equation*}
\end{solution}


\begin{problem}
Solve the given initial value problem using the method of Laplace transforms
\begin{equation*}
y^{\prime \prime} + 3 t y^{\prime} - 6y = 1 ; \quad y(0)=0, \quad y^{\prime}(0) = 0.
\end{equation*}
\end{problem}

\begin{solution}

  First, perform the Laplace Transform of the original IVP. Using formulas
  \begin{eqnarray*}
    \mathcal{L}\{t^{n}f(t)\} & = & (-1)^{n}\frac{d^{n}}{ds^{n}} F(s) \\
    \mathcal{L}\{y^{\prime}(t)\} & = & sY(s) - y(0)
  \end{eqnarray*}
  we write:
  \[ s^2 Y(s) - 3 \frac {d} {ds} \left( s Y(s) \right) - 6Y(s) = \frac {1} {s} \]
  or taking the derivative and rearranging:
  \begin{eqnarray*}
    s^2 Y(s) -3 Y(s) - 3 s Y^{\prime}(s) - 6 Y(s) & = & \frac {1} {s} \\
    -3 s Y^{\prime}(s) - \left( s^2 - 9 \right) Y(s) & = & \frac {1} {s} \\
    Y^{\prime}(s) - \left( \frac {s} {3} - \frac {3} {s} \right) Y(s) & = & - \frac {1} {3 s^2}
  \end{eqnarray*}
  This is a first order linear equation that can be solved using the integrating
  factor:
  \begin{eqnarray*}
    \mu & = & exp\left( - \int \left( \frac {s} {3} - \frac {3} {s} \right) ds \right) \\
    \mu & = & exp\left( - \frac {s^2} {6} + 3 ln(s) \right) \\
    \mu & = & s^3 e^{-\frac {s^2} {6}}
  \end{eqnarray*}
  so that
  \begin{eqnarray*}
    \frac {d} {ds} \left( \mu Y(s) \right) & = & - \frac {s} {3} e^{-\frac {s^2} {6}} \\
    \mu Y(s) & = & - \frac {1} {3} \int s e^{-\frac {s^2} {6}} ds \\
    \mu Y(s) & = & \int e^{-\frac {s^2} {6}} d\left( - \frac {s^2} {6}  \right) \\
    \mu Y(s) & = & e^{-\frac {s^2} {6}} + C
  \end{eqnarray*}
  We find $Y(s)$:
  \[ Y(s) = \frac {1} {s^3} + \frac {C} {s^3} e^{\frac {s^2} {6}} \]
  Since $Y(s)$ is the LT of $y(t)$ (we assume that $y(t)$ is a 
  piecewise continuous function of exponential order), we must have
  $ \lim_{s \to \infty} Y(s) =0$, $C=0$. Now $Y(s) = \frac {1} {s^3}$ and we find
  the inverse LT:
  \begin{equation*}
    \boxed{y(t) = \frac {t^2} {2}}
  \end{equation*}
\end{solution}




% \begin{problem}
% Find solutions to the given initial value problem
% \begin{equation*}
% y^{\prime \prime} +ty^{\prime}-y = 0 ; \quad y(0)=0, \quad y^{\prime}(0) = 3,
% \end{equation*}
% \end{problem}
% \begin{solution}
% Applying a Laplace transform on both sides, we obtain:
% 
% 
% \begin{equation*}
% {\cal L}\{ y^{\prime \prime} \} +{\cal L}\{ ty^{\prime} \}- {\cal L}\{ y \} = 0
% \end{equation*}
% \begin{equation*}
% \left(s^2 Y(s) - s y(0)-y^{\prime}(0)\right) -\dfrac{{\rm d}}{{\rm d}s} {\cal L}\{y^{\prime} \} -  Y(s) =0
% \end{equation*}
% 
% \begin{equation*}
% \left(s^2 Y(s) - s y(0)-y^{\prime}(0)\right) -\dfrac{{\rm d}}{{\rm d}s} \left( sY(s)-y(0)\right) -  Y(s) =0
% \end{equation*}
% 
% \begin{equation*}
% s^2 Y(s) -3 - sY^{\prime}(s) - Y(s) -  Y(s) =0
% \end{equation*}
% 
% We can rearange this equation in the following way:
% \begin{equation*}
% Y^{\prime}(s) +\left(\dfrac{2}{s}-s\right) Y(s) =-\dfrac{3}{s}
% \end{equation*}
% 
% This is a first order ODE for $Y^{\prime}(s)$.
% The integrating factor for this ODE is
% 
% \begin{equation*}
% I(s) = \exp\left(\int \frac{2}{s}-s {\rm d} s\right) = \exp\left(2\ln(s) -\dfrac{s^2}{2} \right) = s^2 e^{-s^2/2}
% \end{equation*}
% Hence:
% \begin{equation*}
% (s^2 e^{-s^2/2}Y(s))^{\prime} =-3 s \, e^{-s^2/2}
% \end{equation*}
% 
% \begin{eqnarray*}
% Y(s) & = & 3\dfrac{e^{s^2/2}}{s^2} \int -s e^{-s^2/2} {\rm d}s + C\dfrac{e^{s^2/2}}{s^2}\\
%  & = &  3\dfrac{e^{s^2/2}}{s^2} e^{-s^2/2} + C\dfrac{e^{s^2/2}}{s^2} \\
%   & = &  \dfrac{3}{s^2} + C\dfrac{e^{s^2/2}}{s^2} \\
% \end{eqnarray*}
% Since we require that $ \lim_{s \to \infty} Y(s) =0$, $C=0$.
% 
% Hence:
% \begin{equation}
% Y(s) =\dfrac{3}{s^2}
% \end{equation}
% Taking the inverse Laplace transform, we obtain:
% 
% \begin{equation}
% \boxed{y(t) = 3{\cal L}^{-1} \left\{\dfrac{1}{s^2} \right\} = 3t}.
% \end{equation}
% \end{solution}



\end{document}
