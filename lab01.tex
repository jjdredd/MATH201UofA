% MATH 201 Lab notes (c) by Carlos Contreras And Philippe Gaudreau
% MATH 201 Lab notes is licensed under a 
% Creative Commons Attribution 4.0 International license.
% CC BY 4.0

% You should have received a copy of the license along with this
% work. If not, see <http://creativecommons.org/licenses/by/4.0/>.

\documentclass[11pt]{article}
% MATH 201 Lab notes (c) by Carlos Contreras And Philippe Gaudreau
% MATH 201 Lab notes is licensed under a 
% Creative Commons Attribution 4.0 International license.
% CC BY 4.0

% You should have received a copy of the license along with this
% work. If not, see <http://creativecommons.org/licenses/by/4.0/>.

%% libraries
\usepackage[utf8x]{inputenc}
\usepackage{xcolor}
\usepackage[left=1.5cm,right=1.5cm,top=2.0cm,bottom=1.5cm,headheight=110pt]{geometry}
\usepackage{amsmath}
\usepackage{amssymb}
\usepackage{graphicx}
\usepackage{xifthen}
\usepackage{sverb}
\usepackage{fancyhdr}
\usepackage{mdframed}
\usepackage{textcomp}

%%%%%%%%%%%%%%%%%%%%%%%%%%%%%%%%%%%%%%%%%%%%%%%%%%%%%%%%%%%%%%%%%%%%%%%
% PDF compiling
\usepackage{ifpdf}
\ifpdf %
        \DeclareGraphicsExtensions{.pdf}%
\else %
        \DeclareGraphicsExtensions{.eps,.ps}%
\fi

%%%%%%%%%%%%%%%%%%%%%%%%%%%%%%%%%%%%%%%%%%%%%%%%%%%%%%%%%%%%%%%%%%%%%%%
% Figures path
\graphicspath{{figures/}}

%%%%%%%%%%%%%%%%%%%%%%%%%%%%%%%%%%%%%%%%%%%%%%%%%%%%%%%%%%%%%%%%%%%%%%%
% Problem counter
\newcounter{Problem}
\setcounter{Problem}{0}

%%%%%%%%%%%%%%%%%%%%%%%%%%%%%%%%%%%%%%%%%%%%%%%%%%%%%%%%%%%%%%%%%%%%%%%
% Definitions
\def\LabSolutions{\clearpage \newpage \begin{center} {\Large \it Solutions} \end{center} \setcounter{Problem}{0}}
\def\QuizSolutions{\newpage \begin{center} {\Large \it Solutions} \end{center} \setcounter{Problem}{0}}
\def\degree{\textdegree}
\def\grade#1{\begin{flushright} {\small [#1]}\\ \end{flushright} \vspace{-10pt}}
\def\codecolor{red!50!black}
\def\code#1{\textcolor{\codecolor}{\tt #1}}
\def\examname#1{%
                \ifnum\value{page}>1%
                    \newpage%
                \else%
                    \vspace*{5pt}%
                \fi%
                \large \textbf{#1} \setcounter{Problem}{0}\vspace{10pt}}
\def\topic#1{\par\needspace{2\baselineskip} \noindent \textsl{\footnotesize #1}}

 
%%%%%%%%%%%%%%%%%%%%%%%%%%%%%%%%%%%%%%%%%%%%%%%%%%%%%%%%%%%%%%%%%%%%%%%
% Environments
\newenvironment{problem}%
     {\stepcounter{Problem}%
      \begin{list}{\textbf{\arabic{Problem}}.~}{}%
      \item}%
     {\end{list}\vspace*{5pt}}

\newenvironment{solution}%
     {\indent \textit{Solution} \newline}%
     {\begin{flushright}$\blacksquare$\end{flushright}}

\newenvironment{preamble}%
     {\vspace*{1em}\begin{mdframed}[leftmargin=1cm,rightmargin=1cm]}%
     {\end{mdframed}\vspace*{1em}}

\newenvironment{multchoice}%
     {\begin{enumerate} \addtolength{\leftskip}{2em} \renewcommand{\labelenumi}{(\alph{enumi})}}
     {\end{enumerate}}

\newenvironment{formulaitem}%
     {\setlength{\leftmargini}{1.5em}\begin{itemize}%
      \setlength\itemindent{-\itemindent}%
      \renewcommand{\labelitemi}{$\rightarrow$}}%
     {\end{itemize}}


%%%%%%%%%%%%%%%%%%%%%%%%%%%%%%%%%%%%%%%%%%%%%%%%%%%%%%%%%%%%%%%%%%%%%%%
% New theorems
\newtheorem{theorem}{Theorem}


\makeatletter

%%%%%%%%%%%%%%%%%%%%%%%%%%%%%%%%%%%%%%%%%%%%%%%%%%%%%%%%%%%%%%%%%%%%%%%
%% New commands
\newcommand*{\course}[1]{\gdef\@course{#1}}
\newcommand*{\coursecode}[1]{\gdef\@coursecode{#1}}
\newcommand*{\term}[1]{\gdef\@term{#1}}
\newcommand*{\instructor}[1]{\gdef\@instructor{#1}}
\newcommand*{\lqnumber}[1]{\gdef\@lqnumber{#1}}
\newcommand*{\labtitle}[1]{\gdef\@labtitle{#1}}
\newcommand*{\quizversion}[1]{\gdef\@quizversion{#1}}
\newcommand*{\probleminfo}[1]{\noindent \textsl{\footnotesize #1}}

% Title header for labs
\newcommand\makelabtitle{%
  \begin{flushleft}%
  {\scshape \@coursecode~\@course~-- University of Alberta}\\%
  {\scshape \@term~-- Labs -- \@instructor}\\%
  {\scshape Authors: Carlos Contreras, Philippe Gaudreau, Rahim Usubov}%
  \end{flushleft}%
  \begin{center}%
  {\Large \bf \@lqnumber:~\@labtitle}%
  \end{center}%
  \thispagestyle{empty}%
  \global\let\@course\@empty%
  \global\let\@labtitle\@empty%
}

% Title header for quizzes
\newcommand\makequiztitle{%
  \begin{flushleft}%
  {\scshape \@coursecode~\@course~-- University of Alberta}\\%
  {\scshape \@term~-- Labs -- \@instructor}\\%
  \end{flushleft}%
  \begin{center}%
  {\Large \bf \@lqnumber} \marginpar{\tiny\tt [\@quizversion]}%
  \end{center}%
  \thispagestyle{empty}%
  \global\let\@course\@empty%
  \global\let\@quizversion\@empty%
}

%%%%%%%%%%%%%%%%%%%%%%%%%%%%%%%%%%%%%%%%%%%%%%%%%%%%%%%%%%%%%%%%%%%%%%%
% Fancy header package
\fancyhead[L]{\small {\scshape \@coursecode~-- \@lqnumber~-- \@term~-- \@instructor}}
\pagestyle{fancy}

\makeatother


\usepackage{hyperref}
\usepackage{cancel}


\begin{document}

\course{Differential Equations}
\coursecode{MATH 201}
\term{Fall 2020}
\instructor{Carlos Contreras, Rahim Usubov}
\lqnumber{Lab 1}
\labtitle{First-order differential equations}
\makelabtitle

% separable
\topic{Separable equations}
\begin{problem}
Determine whether the given differential equation is separable
\begin{equation*}
     (xy^2 + 3y^2) dy - 2x \, dx =0.
\end{equation*}
\end{problem}

% % separable
% \begin{problem}
% Solve the equation
% \begin{equation*}
%      \dfrac{ dx }{dt} = 3 x t^2.
% \end{equation*}
% \end{problem}

% separable
\begin{problem}
Solve the equation
\begin{equation*}
     \dfrac{ dy }{dx} = \frac{x}{y^2 \sqrt{1+x}}
\end{equation*}
\end{problem}



% separable
\begin{problem}
Solve the following differential equation
\begin{equation*}
     \dfrac{ ds }{dt}  =  t \ln(s^{2t}) + 8 t^2.
\end{equation*}
\end{problem}



% separable implicit
\begin{problem}
Solve the equation
\begin{equation*}
     \dfrac{ dy }{dx} = \frac{\sec^2(y)}{1+x^2}.
\end{equation*}
\end{problem}

% separable ivp
\begin{problem}
Solve the initial value problem
\begin{equation*}
     \dfrac{1}{\theta} \dfrac{ dy }{d\theta} = \frac{y \sin(\theta)}{y^2+1}, \quad y(\pi)=1.
\end{equation*}
\end{problem}





% integrating factor
\topic{Linear equations}
\begin{problem}
Obtain the general solution to the equation
\begin{equation*}
      \dfrac{ dy }{dx} = \frac{y }{x} + 2 x + 1.
\end{equation*}
\end{problem}


% integrating factor
\begin{problem}
Obtain the general solution to the equation
\begin{equation*}
      x \dfrac{ dy }{dx}+ 2 y = x^{-3}.
\end{equation*}
\end{problem}


% integrating factor
\begin{problem}
Obtain the general solution to the equation
\begin{equation*}
     (t+y+1)dt -  dy =0 .
\end{equation*}
\end{problem}

% integrating factor
\begin{problem}
Solve the initial value problem
\begin{equation*}
     t^2 \dfrac{ dx }{dt} + 3 t x = t^4 \ln(t) +1 , \quad x(1) =0. 
\end{equation*}
\end{problem}


% homogeneous 1st order
\topic{1st order homogeneous}
\begin{problem}
Solve the following problem
\begin{equation*}
\frac{dy}{dx}=\frac{x^{2}+3y^{2}}{2xy}, \quad x>0. 
\end{equation*}
\end{problem}


% homogeneous 1st order
\begin{problem}
  Solve the following equation:
  \begin{equation*}
    \begin{aligned}
      \left( x^2 + 3x y + y^2 \right)dx - x^2 dy = 0, \quad x>0.
    \end{aligned}
  \end{equation*}
\end{problem}



% exact equation
\topic{Exact equations}
\begin{problem}
Solve the following problem
\begin{equation*}
2xy dx+(x^{2}-y^{2})dy=0. 
\end{equation*}
\end{problem}


\begin{problem}
Solve the following problem
\begin{equation*}
ydx + (2x - y e^{y}) dy = 0.
\end{equation*}
\end{problem}


% non-exact
\begin{problem}
Solve the following problem
\begin{equation*}
\left( x + 2 \right) sin(y) + x cos(y) y^{\prime} = 0.
\end{equation*}
\end{problem}



% Bernoulli
\topic{Bernoulli}
\begin{problem}
Solve the following problem
\begin{equation*}
     \frac{dy}{dx}=\frac{(1+x)y-6y^{3}}{2x}.
\end{equation*}
\end{problem}




\begin{problem}
Solve following problem
\begin{equation*}
      y'+3y'=4xy^{3}.
\end{equation*}
\end{problem}


\begin{problem}
Solve following IVP
\begin{equation*}
      xe^{x}y' = (x-1)e^{x}y+y^{2}, \quad y(1)=e.
\end{equation*}
\end{problem}




%%%%%%%%%%%%%%%%%%%%%%%%%%%%%%%%%%%%%%%%%%%%%%%%%%%%%%%%%%%%%%%%%%%%%%%%%%%%%%%%%%%%%%%%%%%%%%%%%%%



\LabSolutions

All problems from: Nagel, Saff \& Sneider, \textit{Fundamentals of Differential Equations}, Eight Edition, Adisson--Wesley.

\begin{preamble}
\textbf{Definitions and formulas}

\begin{formulaitem}
\item A \textbf{first-order linear equation} has the form
\[y' +P(x)y=Q(x).\]
We solve this equations using the \textbf{integrating factor} formula
\[I(x)=e^{\int P(x) dx},\qquad y(x)=\frac{1}{I(x)}\left[\int I(x)Q(x)dx +C \right].\]

\item If the right hand side of a differential equation can be expressed as a function of the ratio $y/x$, \[\frac{dy}{dx}=f(x,y)=g\left(\frac{y}{x}\right),\] then the equation is called \textbf{first order homogeneous}. In such case, we apply the change of variable $v=y/x$ to make the equation separable.

\item Differential equations of the form
\[M(x,y)dx + N(x,y) dy = 0,\]
are called \textbf{exact equations} if and only if
\[\frac{\partial M}{\partial y}(x,y)=\frac{\partial N}{\partial x}(x,y).\]
If it is not exact we might be able to use an integrating factor:
\begin{enumerate}
 \item If $\frac{M_{ y} - N_{ x}}{ N}$ depends only on $x$ , use $I ( x ) = e^{\int \frac{M_{ y} - N_{ x}}{ N} dx}$ .
 \item If $\frac{N_{ x} - M_{ y}}{ M}$ depends only on $y$ , use $I ( y ) = e^{\int \frac{N_{ x} - M_{ y}}{ M} dx}$ .
\end{enumerate}


\item Differential equations of the form
\[y'+P(x)y=Q(x)y^{p},\quad p\neq 1\]
are called \textbf{Bernoulli equations}. In such case, we apply the change of variable $u=y^{1-p}$ to make the equation first order linear.
\end{formulaitem}
\end{preamble}



\begin{problem}
Determine whether the given differential equation is separable
\begin{equation}
     (xy^2 + 3y^2) dy - 2x \, dx =0.
\end{equation}
\end{problem}
\begin{solution}
This equation is separable since : 
\begin{eqnarray*}
0 & = & (xy^2 + 3y^2) {\rm d}y - 2x \, dx \\
&= & y^2(x + 3) {\rm d}y - 2x \, dx \\
\end{eqnarray*}
Dividing through by $x+3$, we obtain:
\begin{equation*}
 0  =  y^2 {\rm d}y -\dfrac{2x}{x+3} dx \\
\end{equation*}
Moving $\dfrac{2x}{x+3} dx$ to the other side, we obtain our desired result:
\begin{equation*}
 \boxed{y^2 {\rm d}y  =  \dfrac{2x}{x+3} dx }.
\end{equation*}
\end{solution}






\begin{problem}
Solve the equation
\begin{equation}
     \dfrac{ dy }{dx} = \frac{x}{y^2 \sqrt{1+x}}
\end{equation}
\end{problem}
\begin{solution}
This equation is separable, since
\begin{eqnarray*}
y^2dy & = & \frac{x}{ \sqrt{1+x}}dx \\
\int y^2dy & = & \int \frac{x}{ \sqrt{1+x}}dx \\
\frac{y^3}{3} & = & \int \frac{x}{ \sqrt{1+x}}dx \\
\end{eqnarray*}
To solve the integral in right hand side, make $t =1 +x$. This implies that ${d}t ={d}x$, hence
\begin{eqnarray*}
\frac{y^3}{3} & = & \int \frac{t-1}{ \sqrt{t}}dt \\
& = & \int t^{1/2} - t^{-1/2}dt\\
& = & \frac{2 t^{3/2}}{3} - 2 t^{1/2} +C\\
& = & \frac{2 (1+x)^{3/2}}{3} - 2 (1+x)^{1/2} +C\\
\end{eqnarray*}
Isolating $y$, we obtain:
\begin{equation*}
\boxed{y(x) = \left[ 2(1+x)^{3/2} - 6(1+x)^{1/2} +C\right]^{1/3}},
\end{equation*}
for some constant $C$.
\end{solution}





\begin{problem}
Determine whether the given differential equation is separable
\begin{equation}
     \dfrac{ ds }{dt}  =  t \ln(s^{2t}) + 8 t^2.
\end{equation}
\end{problem}

\begin{solution}
This equation is separable since : 
\begin{eqnarray*}
\dfrac{ ds }{dt}  & = &  t \ln(s^{2t}) + 8 t^2 \\
& = & t(2t)\ln(s) + 8 t^2 \\
& = & 2t^2\ln(s) + 8 t^2 \\
& = & 2t^2(\ln(s) + 4 ) \\
\end{eqnarray*}
\begin{equation*}
\Rightarrow \boxed{ \dfrac{ ds }{\ln(s) + 4 } = 2t^2 dt}.
\end{equation*}
\end{solution}




% \begin{problem}
% Solve the equation
% \begin{equation}
%      \dfrac{ dx }{dt} = 3 x t^2.
% \end{equation}
% \end{problem}
% \begin{solution}
% This equation is separable. Putting all the $t$ terms on one side and all the $x$ terms on the other side, we obtain:
% \begin{equation*}
% \dfrac{ dx }{x} =  3 t^2 dt
% \end{equation*}
% Applying the integral operator on both side, we get:
% \begin{eqnarray*}
% \int \dfrac{ dx }{x} & = & \int  3 t^2 dt \\
% \ln |x|  & = & t^3 + C \\
% |x|  & = & e^{t^3 + C}
% \end{eqnarray*}
% \begin{equation*}
%  \Rightarrow \boxed{x(t) = Ae^{t^3}},
% \end{equation*}
% 
% for some constant $A$.
% \end{solution}




\begin{problem}
Solve the equation
\begin{equation}
     \dfrac{ dy }{dx} = \frac{\sec^2(y)}{1+x^2}.
\end{equation}
\end{problem}
\begin{solution}
This equation is separable,
\begin{eqnarray*}
\frac{dy }{\sec^2(y)}& = & \frac{dx}{1+x^2} \\
\cos^2(y)dy  & = & \frac{dx}{1+x^2} \\
\int \cos^2(y)dy  & = & \int \frac{dx}{1+x^2} \\
\int \frac{\cos(2y)}{2}+ \frac{1}{2} \, dy  & = & \arctan(x)+C \\
\frac{\sin(2y)}{4}+ \frac{y}{2}   & = & \arctan(x)+C.
\end{eqnarray*}
In this case, we cannot isolate $y$, hence our solution is given implicitly by
\begin{equation*}
\boxed{\sin(2y(x))+ 2y(x)  =  4\arctan(x)+C}.
\end{equation*}
\end{solution}




\begin{problem}
Solve the initial value problem
\begin{equation}
     \dfrac{1}{\theta} \dfrac{ dy }{d\theta} = \frac{y \sin(\theta)}{y^2+1}, \quad y(\pi)=1.
\end{equation}
\end{problem}
\begin{solution}
This equation is separable,
\begin{eqnarray*}
\frac{y^2+1}{y} dy & = & \theta \sin(\theta) d\theta \\
\int \frac{y^2+1}{y} dy & = &  \int \theta \sin(\theta) d\theta \\
\int y + y^{-1} dy & = &  \int \theta \sin(\theta) d\theta \\
\frac{y^2}{2} + \ln|y| & = &  -\theta \cos(\theta) + \sin(\theta)+C.
\end{eqnarray*}
Setting $y(\pi)=1$, we can solve for $C$.
\begin{eqnarray*}
\frac{1^2}{2} + \ln|1| & = &  -\pi \cos(\pi) + \sin(\pi)+C \\
\frac{1}{2}  & = &  \pi  + C \\
C  & = & \frac{1}{2} - \pi   \\
\end{eqnarray*}
In this case, we cannot isolate $y$, hence our solution is given implicitly by
\begin{equation*}
\boxed{\frac{y(\theta)^2}{2} + \ln|y(\theta)|  =   -\theta \cos(\theta) + \sin(\theta)+\frac{1}{2} - \pi}.
\end{equation*}
\end{solution}

\begin{solution}
We can alternatively use the initial value in the integration, where we compute a \textit{definite integral}. \\
For example, in the previous solution we can integrate between the given initial point $\pi$, and the independent variable $\theta$ (note that we need to use a different variable in the integrand of the second integral)
\begin{eqnarray*}
\int_{\pi}^{\theta} y + y^{-1} dy & = &  \int_{\pi}^{\theta} t \sin(t) dt \\
\Rightarrow \left. \frac{1}{2}y^{2} \right|_{\pi}^{\theta} + \left. \ln |y| \right|_{\pi}^{\theta} & = &\left. -t \cos t \right|_{\pi}^{\theta} + \left. \sin t \right|_{\pi}^{\theta} \\
\Rightarrow \frac{1}{2} y^{2}(\theta) - \frac{1}{2} +\ln |y(\theta)| - 0 & = & -\theta \cos t + \cancelto{{\scriptstyle -1}}{ \pi \cos \pi} + \sin \theta - \cancelto{\scriptstyle 0}{\sin \pi} ,
\end{eqnarray*}
Thus, the implicit solution is,
\begin{equation*}
\boxed{ \frac{1}{2}y^{2}(\theta) -\ln |y(\theta)| = -\theta \cos\theta + \sin \theta +\frac{1}{2} -\pi }.
\end{equation*}

\end{solution}










\begin{problem}
Obtain the general solution to the equation
\begin{equation}\label{formula: question 7}
      \dfrac{ dy }{dx} = \frac{y }{x} + 2 x + 1.
\end{equation}
\end{problem}
\begin{solution}
This equation is not separable, but can be written in the standard form
\begin{equation}\label{formula: special form}
y^{\prime} + P(x) y = Q(x).
\end{equation}
When written in standard form, the integrating factor is given by
\begin{equation}\label{formula: Integrating factor}
I(x) = e^{\int P(x) dx},
\end{equation}
and the solution to any first order non separable equation of the form \eqref{formula: special form} is given by
\begin{equation}\label{formula: Solution}
y(x) = \frac{1}{I(x)} \left[ \int I(x) Q(x) dx + C \right].
\end{equation}
Rewriting equation \eqref{formula: question 7} in the form of equation \eqref{formula: special form}, we obtain
\begin{equation*}
 \dfrac{ dy }{dx} - \frac{1 }{x} y   =  2 x + 1.
\end{equation*}
Hence the integrating factor is
\begin{eqnarray*}
I(x) & = & e^{\int - \frac{1 }{x} dx} \\
& = & e^{-\ln|x|} \\
& = & e^{\ln|x^{-1}|} \\
& = & x^{-1}.
\end{eqnarray*}
Thus, the solution to equation \eqref{formula: question 7} is given by equation \eqref{formula: Solution}
\begin{eqnarray*}
y(x) & = & \frac{1}{x^{-1}} \left[ \int x^{-1} (2x+1) dx + C \right]\\
& = & x \left[ \int (2+x^{-1}) dx + C \right]\\
& = & x \left[ 2x + \ln|x| + C \right]\\
& = &  2x^2 + x\ln|x| + Cx.
\end{eqnarray*}
Finally, for some constant $C$ the general solution is,
\begin{equation*}
     \boxed{y(x) = 2x^2 + x\ln|x| + Cx }.
\end{equation*}
\end{solution}

\begin{solution}
Let's say we forgot the formula \eqref{formula: Solution} for the integrating factor. The two things we need to remember is that the equation can be written in the special form \eqref{formula: special form} (emphasizing on what multiplies $y$)
\begin{equation*}
y^{\prime} + P(x) y = Q(x),
\end{equation*}
and that we need to compute $e^{\int P(x)dx}$, so that,
\begin{equation*}
     \frac{d(e^{\int P(x)dx} y)}{dx} = e^{\int P(x)dx} y' + e^{\int P(x)dx}P(x) y.
\end{equation*}
Now, 
\begin{eqnarray*}
I(x) & = & e^{\int - \frac{1 }{x} dx} \\
& = & e^{-\ln|x|} \\
& = & e^{\ln|x^{-1}|} \\
& = & x^{-1}. \\
\end{eqnarray*}
If we multiply $x^{-1}$ on both sides of the equation we get
\begin{gather*}
     y' - x^{-1}y = (2x +1)\\
     \Rightarrow x^{-1}y' - x^{-2}y = x^{-1}(2x +1)\\
     \Rightarrow (x^{-1}y)'=2 + x^{-1}\\
     \Rightarrow \int d(x^{-1}y) = \int (2 +x^{-1})dx\\
     \Rightarrow x^{-1} y = 2x +\ln |x| + C,
\end{gather*}
Thus, the solution for some constant $C$ is, 
\begin{equation*}
     \boxed{y(x) = 2 x^{2} +x\ln |x| +C x}.
\end{equation*}

\end{solution}

\textbf{Note:}
The general idea of the integrating facor method(s) is to transform the
term $y^{\prime} + P(x) y$ into a full derivative using the product rule,
with the help of the integrating factor $I(x)$. Suppose that we multiply the
expression $y^{\prime} + P(x) y$ by some function (the integrating factor)
$I(x)$:
\[ I(x) y^{\prime} + I(x) P(x) y \],
to transform this into a full derivative $\left( y I(x) \right)^{\prime}$.
For this, we must require (according to the product rule):
\[ I^{\prime}(x) = I(x) P(x) \]
We can use the separation of variables to solve this differential equation
and find $I(x)$:
\[ I(x) = e^{\int P(x)dx}\]





\begin{problem}
Obtain the general solution to the equation
\begin{equation}
      x \dfrac{ dy }{dx}+ 2 y = x^{-3}.
\end{equation}
\end{problem}
\begin{solution}
Rewriting this equation in the form of equation \eqref{formula: special form}, we obtain
\begin{equation*}
\dfrac{ dy }{dx}+ \frac{2}{x} y = x^{-4}.
\end{equation*}
The integrating factor is given by 
\begin{eqnarray*}
I(x) & = & e^{\int  \frac{2 }{x} dx} \\
& = & e^{2\ln|x|} \\
& = & e^{\ln|x^{2}|} \\
& = & x^{2}.
\end{eqnarray*}
Hence, the general solution is given by equation \eqref{formula: Solution}.
\begin{eqnarray*}
y(x) & = & \frac{1}{x^{2}} \left[ \int x^{2} x^{-4} dx + C \right]\\
& = & x^{-2} \left[ \int x^{-2} dx + C \right]\\
& = & x^{-2} \left[ -x^{-1} + C \right]\\
& = &  -x^{-3} + C x^{-2}.
\end{eqnarray*}
Hence, for some constant $C$ the general solution is,
\begin{equation*}
     \boxed{y(x) = -\dfrac{1}{x^{3}} + \dfrac{C}{x^{2}}}.
\end{equation*}
\end{solution}




\begin{problem}
Obtain the general solution to the equation
\begin{equation}
     (t+y+1)dt -  dy =0 .
\end{equation}
\end{problem}
\begin{solution}
Rewriting this equation in the form of equation \eqref{formula: special form}, we obtain
\begin{equation*}
\dfrac{ dy }{dt} - y= t+1.
\end{equation*}
The integrating factor is given by 
\begin{eqnarray*}
I(t) & = & e^{\int  -1 dt} \\
& = & e^{-t}.
\end{eqnarray*}
The general solution is given by equation \eqref{formula: Solution}
\begin{eqnarray*}
y(x) & = & \frac{1}{e^{-t}} \left[ \int e^{-t}(t+1) dx + C \right]\\
& = & e^{t}  \left[ -(t+2)e^{-t} + C \right]\\
& = & -(t+2) + C e^{t}.
\end{eqnarray*}
Hence, for some constant $C$ the general solution is, 
\begin{equation*}
     \boxed{y(x) = -(t+2) + C e^{t} }.
\end{equation*}

\end{solution}





\begin{problem}
Solve the initial value problem
\begin{equation}
     t^2 \dfrac{ dx }{dt} + 3 t x = t^4 \ln(t) +1 , \quad x(1) =0 .
\end{equation}
\end{problem}
\begin{solution}
Rewriting this equation in the form of equation \eqref{formula: special form}, we obtain:

\begin{equation*}
\dfrac{ dx }{dt} + \frac{3}{t} x = t^2 \ln(t) + t^{-2}
\end{equation*}
The integrating factor is given by 
\begin{eqnarray*}
I(t) & = & e^{\int  \frac{3 }{t} dt} \\
& = & e^{3\ln|t|} \\
& = & e^{\ln|t^{3}|} \\
& = & t^{3} .
\end{eqnarray*}
The general solution is given by equation \eqref{formula: Solution}
\begin{eqnarray*}
x(t) & = & \frac{1}{t^{3}} \left[ \int t^{3}(t^2 \ln(t) + t^{-2}) dt + C \right]\\
& = & t^{-3} \left[ \int t^5 \ln(t) + t dt + C \right]\\
& = & t^{-3} \left[ \frac{t^6}{6} \ln(t) - \frac{t^6}{36} + \frac{t^2}{2} + C \right]\\
& = &  \frac{t^3}{6} \ln(t) - \frac{t^3}{36} + \frac{1}{2t} + \frac{C}{t^3}.
\end{eqnarray*}
Using the initial condition $x(1) =0$, we can solve for $C$
\begin{eqnarray*}
x(1) = 0 & = & \frac{(1)^3}{6} \ln(1) - \frac{(1)^3}{36} + \frac{1}{2(1)} + \frac{C}{(1)^3} \\
& = &  - \frac{1}{36} + \frac{1}{2} + C \\
 \Rightarrow C & = & -\dfrac{17}{36}.
\end{eqnarray*}
Hence, the solution is
\begin{equation*}
     \boxed{x(t) = \dfrac{t^3}{6} \ln(t) - \dfrac{t^3}{36} + \dfrac{1}{2t} - \dfrac{17}{36t^3} }.
\end{equation*}

\end{solution}



% homogeneous 1st order
\begin{problem}
Solve the following problem
\begin{equation*}
\frac{dy}{dx}=\frac{x^{2}+3y^{2}}{2xy}, \quad x>0. 
\end{equation*}
\end{problem}
\begin{solution}
Note that this equation is not separable, however the change of variable $u=y/x$ makes the equation separable. In such case,
$$y(x)=x u(x) \,\,\text{ and }\,\, y'=u + x u'.$$ Rewrite the differential equation as (the trick is to take the appropriate common factor in the numerator and denominator)
\begin{equation}\label{equ:odehomo1}
     \frac{du}{dx}=\frac{\cancel{x^{2}}(1+3\left(\frac{y}{x}\right)^{2})}{2\cancel{x^{2}}\left(\frac{y}{x}\right)}=\frac{1+3\left(\frac{y}{x}\right)^{2}}{2\left(\frac{y}{x}\right)},
\end{equation}
so the suggested change of variable is now natural. \\
Equations of the form $dy/dx = f(x,y)$, where $f(x,y)$ can be expressed as a function of the ratio $y/x$ are called \textit{homogeneous}. Using the change of variable in \eqref{equ:odehomo1} we have,
\begin{eqnarray*}
u +x \frac{du}{dx}& = & \frac{1+3u^{2}}{2u}\\
x\frac{du}{dx} & = & \frac{1+3u^2 -2u^2}{2u}\\
x\frac{du}{dx}& = & \frac{1+u^{2}}{2u}\\
\frac{2udu}{1+u^{2}}& = &\frac{dx}{x}. 
\end{eqnarray*}
Which is a separable equation. For the first integral we use the change of variable $v=1+u^{2}$. Hence,
\begin{eqnarray*}
\ln |1+u^{2}|& = & \ln |x| + C\\
1+u^2& = & C x\\
u & = &\sqrt{C x - 1},
\end{eqnarray*}
Finally, replacing the change of variable, the solution for some constant C is,
\begin{equation*}
\boxed{y(x)=x\sqrt{C x - 1}}.
\end{equation*}

\end{solution}



% Rahim, another example of homogenous equation
\begin{problem}
  Solve the following equation:
  \begin{equation}
    \begin{aligned}
      \left( x^2 + 3x y + y^2 \right)dx - x^2 dy = 0, \quad x>0.
    \end{aligned}
  \end{equation}
\end{problem}

\begin{solution}
  First, it's convenient to rewrite the equation:
  \begin{eqnarray*}
    x^2 dy & = & \left( x^2 + 3x y + y^2 \right) dx \\
    \frac {dy} {dx} & = & 1 + 3\left( \frac {y} {x} \right)  + \left( \frac {y} {x} \right)^2
  \end{eqnarray*}
  Now note that the right hand side depends only on $u(x) = \left( \frac {y} {x} \right)$,
  so we can make a substitution
  \[ y(x) = x u(x) \]
  From this substitution:
  \[ \frac {dy} {dx} = x \frac {du} {dx} + u \]
  We can use it in the left hand side instead of $y^{\prime}$
  \begin{eqnarray*}
    x \frac {du} {dx} + u & = & 1 + 3 u + u^2 \\
    x \frac {du} {dx} & = & 1 + 2 u + u^2 \\
    \frac {du} {1 + 2u + u^2} & = & \frac {dx} {x}
  \end{eqnarray*}
  We see that the original equation is now transformed into a seperable equation.
  Now we can simplify and integrate both sides to find $u(x)$
  \begin{eqnarray*}
    \int \frac {du} {(1 + u)^2} & = & \int \frac {dx} {x} \\
    - \frac {1} {1 + u} + C & = & ln|x| \\
    \frac {1} {1 + u} & = & C - ln|x| \\
    u(x) & = & \frac {1} {C - ln|x|} - 1,
  \end{eqnarray*}
  and finally we substitute back $y(x)$ to find the solution
  \[ y(x) = \frac {x} {C - ln(x)} - x \]
\end{solution}


% exact equation
\begin{problem}
Solve the following problem
\begin{equation*}
2xy dx+(x^{2}-y^{2})dy=0. 
\end{equation*}
\end{problem}
\begin{solution}
This equation is exact since,
\[M(x,y)=2xy,\,\, N(x,y)= x^{2}-y^{2} \Rightarrow \partial_{y}M(x,y)= 2x = \partial_{x}N(x,y). \]
Thus, 
\[F(x,y)= \int M(x,y) dx = x^{2}y + g(y),\]
and
\[\partial_{y}F(x,y)=x^{2}+g'(y) = x^{2}-y^{2} = N(x,y) \Rightarrow g'(y) = -y^{2} \Rightarrow g(y)= -\frac{1}{3}y^{3}-C.\]
Hence,
\[\boxed{F(x,y)=x^{2}y-\frac{1}{3}y^{3}=C}\]
\end{solution}


\begin{problem}
Solve the following problem
\begin{equation*}
ydx + (2x - y e^{y}) dy = 0.
\end{equation*}
\end{problem}
\begin{solution}
Note that the equation is not separable and not exact
\[M _{y} = 1 \neq N_{ x} = 2 .\]
To find the integrating factor we first compute either
\[\frac {M_{ y} - N_{ x}} {N} ,\]
to create an integrating factor depending only on $x$ (i.e., $\mu( x )$), or
\[\frac{N_{ x} - M_{ y}}{ M} ,\]
to create an integrating factor depending only on $y$ (i.e., $\mu ( y )$).
Take, for example,
\[\frac{N_{ x} - M_{ y}}{ M} = \frac{2 - 1}{ y} = \frac{1}{y}.\]
Since this function depends only on $y$ we can define the following integrating factor 
\[\mu( y ) = e^{ \int \frac{N_{ x} - M_{ y}}{ M} dy }= e^{ \int \frac{ dy}{ y}} = y.\]
Multiplying both sides of the equation by $\mu ( y ) = y$ , we get
\[y^{ 2} dx + (2 xy - y^{ 2} e^{ y} ) dy = 0 .\]
Note that this new equation is exact.  Now we apply the method for exact equations
\[F ( x,y ) = \int M ( x,y ) \partial x = Z\int y^{ 2} \partial x = xy 2 + h ( y ) .\]
and
\[\partial_{ y} F ( x,y ) = 2 xy + h'( y ) = 2 xy - y^{ 2} e^{ y} = N ( x,y ) \Rightarrow h'( y ) = - y^{ 2} e^{ y} ,\]
which after integration by parts gives,
\[ h ( y ) = ( - y^{ 2} + 2 y - 2) e^{y} +C.\]
Finally,
\[\boxed{F ( x,y ) = xy^{ 2} + ( - y^{ 2} + 2 y - 2) e^{ y} = C }.\]
\end{solution}



\begin{problem}
Solve the following problem
\begin{equation*}
\left( x + 2 \right) sin(y) + x cos(y) y^{\prime} = 0.
\end{equation*}
\end{problem}

\begin{solution}
  Here $M(x, y) = \left( x + 2 \right) sin(y)$ and $N(x, y) = x cos(y)$. The equation
  is not exact. We need to find the integrating factor: a function $\mu(x, y)$ that
  we multiply by the whole equation:
  \[ \mu M dx + \mu N dy = 0 \]
  requiring that it is of exact type. Formally, we require
  \[ M \mu_y - N \mu_x + \left( M_y - N_x \right) \mu = 0 \].
  This is a PDE that's not trivial to solve, but in some cases we can make
  simplifying assumptions, like $\mu = \mu(x)$ or $\mu = \mu(y)$ in the former case
  we have
  \begin{eqnarray*}
    & -N \mu_x + \left( M_y - N_x \right) \mu = 0 \\
    & \mu^{\prime} = \frac {M_y - N_x} {N} \mu.
  \end{eqnarray*}
  In this case, since $\mu = \mu(x)$, the expression in the right hand side
  must also be a function of only $x$. If it's not the case we can try the second
  assumption $\mu = \mu(y)$.
  In our case
  \[ \frac {M_y - N_x} {N} = \frac {\left( x + 2 \right) cos(y) - cos(y)} {x cos(y)}
    = \frac {x+1} {x} \]
  is a function of only $x$ so we may continue to find the integrating factor $\mu$.
  We have a seperable equation for $\mu(x)$:
  \begin{eqnarray*}
    \frac {x+1} {x} dx & = & \frac {d \mu} {\mu} \\
    \mu & = & x e^x
  \end{eqnarray*}
  Once we multiply the original equation by the integrating factor we get an exact
  equation:
  \[ \left( x + 2 \right) x e^x sin(y) dx + x^2 e^x cos(y) dy = 0. \]
  Now we can solve it as an exact equation:
  \begin{eqnarray*}
    & \frac {\partial F} {\partial x} = \left( x + 2 \right) x e^x sin(y) \\\\
    & F = \int \left( x + 2 \right) x e^x sin(y) dx + g(y) = x^2 e^x sin(y) + g(y) \\\\
    & \frac {\partial F} {\partial y} = x^2 cos(y) e^x
  \end{eqnarray*}
  From this we conclude that
  \begin{eqnarray*}
    x^2 e^x cos(y) + g^{\prime} & = & x^2 e^x cos(y)\\
    g(y) & = & C
  \end{eqnarray*}
  The solution is given implicitly:
  \[ x^2 e^x sin(y) = C \]
  
  
\end{solution}



% Bernoulli equation
\begin{problem}
Solve the following problem
\begin{equation*}
     \frac{dy}{dx}=\frac{(1+x)y-6y^{3}}{2x}.
\end{equation*}
\end{problem}
\begin{solution}
Write the equation in the standard form
\[y'-\frac{1}{2}\left(\frac{1}{x}+1\right)y=-\frac{3}{x}y^{3},\]
which is a Bernoulli equation. The change of variable is then,
\[u = y^{1-3}=y^{-2}\Rightarrow u'=-2 y^{-3}y' \,\,\text{ and }\,\, y = \pm u^{-1/2}.\]
Dividing the equation over $y^{3}$ (always divide over $y^{p}$ to get $u'$) and using the substitution
\begin{gather*}
y^{-3}y'-\frac{1}{2}\left(\frac{1}{x}+1\right)y^{-2}=-\frac{3}{x} \\
-\frac{1}{2}u' - \frac{1}{2} \left(\frac{1}{x}+1\right)u=-\frac{3}{x}. \\
u' + \left(\frac{1}{x}+1\right)u=\frac{6}{x}.
\end{gather*}
This equation can be solved using integrating factor,
\[I(x)=e^{\int\left(\frac{1}{x}+1\right)dx}=e^{\ln|x|+x}=xe^{x},\]
then
\[u(x)= \frac{1}{x e^{x}}\left[\int xe^{x}\frac{6}{x}dx + C\right]= \frac{6}{x}+\frac{C}{xe^{x}}.\]
Going back to $y(x)$,
\[\boxed{y(x)= \pm\left(\frac{6}{x}+\frac{C}{xe^{x}}\right)^{-1/2}}.\]
\end{solution}



% \begin{problem}
% Solve following problem
% \begin{equation*}
%       y'+3y'=4xy^{3}.
% \end{equation*}
% \end{problem}
% \begin{solution}
% Write the equation in the standard form
% \[y'+3y=4xy^{3},\]
% which is a Bernoulli equation. The change of variable is then,
% \[u = y^{1-3}=y^{-2}\Rightarrow u'=-2 y^{-3}y' \,\,\text{ and }\,\, y = \pm u^{-1/2}.\]
% Dividing the equation over $y^{3}$ (always divide over $y^{p}$ to get $u'$) and using the substitution
% \begin{gather*}
% y^{-3}y'+3y^{-2}=4x \\
% -\frac{1}{2}u' +3u=4{x}. \\
% u' - 6 u= -8x.
% \end{gather*}
% This equation can be solved using integrating factor,
% \[I(x)=e^{\int-6 dx}=e^{-6x},\]
% then
% \[u(x)= \frac{1}{e^{-6x}}\left[\int e^{-6x}(-8)xdx + C\right]= \frac{4}{3}x+\frac{4}{9}+Ce^{6x}.\]
% Going back to $y(x)$,
% \[\boxed{y(x)= \pm\left(\frac{4}{3}x+\frac{4}{9}+Ce^{6x}\right)^{-1/2}}.\]
% \end{solution}




\begin{problem}
Solve following problem
\begin{equation*}
      xe^{x}y'= (x-1)e^{x}y+y^{2}
\end{equation*}
\end{problem}
\begin{solution}
Write the equation in the standard form
\[y'+\left( \frac{1}{x} - 1 \right)y=\frac{e^{-x}}{x}y^{2},\]
which is a Bernoulli equation. The change of variable is,
\[u = y^{1-2}=y^{-1}\Rightarrow u'=- y^{-2}y' \Rightarrow -u'= y^{-2}y' \,\,\text{ and }\,\, y = \pm u^{-1}.\]
Dividing the equation over $y^{2}$ (always divide over $y^{p}$ to get $u'$) and using the substitution
\begin{gather*}
y^{-2}y'+\left( \frac{1}{x} - 1 \right)y^{-1}=\frac{e^{-x}}{x} \\
\Rightarrow -u' + \left( \frac{1}{x} - 1 \right)u=\frac{e^{-x}}{x} \\
\Rightarrow u' + \left( 1 - \frac{1}{x} \right) u= -\frac{e^{-x}}{x}.
\end{gather*}
This equation can be solved using integrating factor,
\[I(x)=e^{\int\left( 1 - \frac{1}{x} \right) dx}=e^{x - \ln |x|}=\frac{e^{x}}{x},\]
then
\[u(x)= xe^{-x}\left[-\int \frac{e^{x}}{x} \frac{e^{-x}}{x} dx + C\right]= xe^{-x}\left[\frac{1}{x} + C \right].\]
Going back to $y(x)$,
\[y(x)= \left( e^{-x}  + Cxe^{-x} \right)^{-1}.\]
Using the initial condition $y(1)=e$,
\[e=\left( e^{-1}  + Ce^{-1} \right)^{-1}\Rightarrow e^{-1}=\left( e^{-1}  + Ce^{-1} \right),\]
which implies $C=0$. Thus, the solution to the IVP is, to our surprise, as simple as
\[\boxed{y(x)= e^{x}}.\]
\end{solution}



\end{document}
